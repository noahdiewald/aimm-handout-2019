\section{Introduction}

\begin{itemize}
\item Potawatomi is a highly endangered Algonquian language spoken in
  Northern Wisconsin.
\item Since at least the 1930s, researchers have consistently
  documented overabundance in noun inflection
  \citep{hockett1948potawatomi2, potdict, lockwood2017potawatomi}.
\item Certain aspects of the overabundant system can be seen in other
  languages with genetic and areal relationships to Potawatomi
  \citep{odawagrammar, ahenakew1987cree}.
\end{itemize}

\section{What is Overabundance?}

\begin{itemize}
\item Overabundance \citep{thornton2011overabundance,
    thornton2012reduction} is when more than one form of a word can
  realize a particular inflection.
\item Overabundant forms are called cellmates.
\end{itemize}

\pex
\a Last night I dreamt that somebody loved me. (Song lyric from {\it Last Night I dreamt that Somebody Loved Me} by the Smiths)
\a Last night I dreamed of chickens, (First line from the poem {\it Last Night I Dreamed of Chickens} by Jack Prelutsky)
\xe

\section{How is Potawatomi Overabundant?}

\begin{itemize}
\item Potawatomi noun inflection shows overabundant behavior according
  to three different criteria:
  \begin{enumerate}
  \item stem shape (Example (\ref{snake})),
  \item the form of prefixes indicating person (Example (\ref{canoe}))
    and
  \item the optionality of a suffix indicating possession (Example
    (\ref{string})).
  \end{enumerate}
\item The last two of these are predicted to affect the majority of
  nouns according to both \citet{hockett1948potawatomi2,
    lockwood2017potawatomi}.
\end{itemize}

\pex\label{snake}
\a\label{snakev}
\begingl
\gla mnedok // 
\glb mnedo-k //
\glc snake-{\sc pl} //
\glft `snakes' //
\endgl

\a\label{snakey}
\begingl
\gla mnedoyek //
\glb mnedoy-k //
\glc snake-{\sc pl} //
\glft `snakes' \trailingcitation{\citep[p. 80]{potdict}} //
\endgl
\xe

\pex\label{canoe}
\a\label{canoeshort}
\begingl
\gla njiman // 
\glb n-jiman //
\glc {\sc 1.pos}-canoe //
\glft `my canoe' //
\endgl

\a\label{canoelong}
\begingl
\gla ndejiman //
\glb ned-jiman //
\glc {\sc 1.pos}-canoe //
\glft `my canoe' \trailingcitation{\citep[p. 69]{hockett1948potawatomi2}} //
\endgl
\xe

\pex\label{string}
\a\label{string-}
\begingl
\gla ndekobjegen // 
\glb n-tekopjegen //
\glc {\sc 1.pos}-string //
\glft `my string' //
\endgl

\a\label{string-m}
\begingl
\gla ndekobjegnem //
\glb n-tekopjegen-m //
\glc {\sc 1.pos}-string-{\sc poss} //
\glft `my string' \trailingcitation{\citep[p. 115]{potdict}} //
\endgl
\xe

\subsection{Stem Form Variation}

Hockett notes a number of stem classes in Potawatomi that can be
determined by various criteria. Three such classes are involved in the
variation seen in (\ref{snake}). Without a suffix, stems in these
classes end in /a,ɛ:,o:/. The classes are delineated according to
whether an epenthetic /y/ occurs between the stem and any suffix.

\begin{enumerate}
\item Class [3] never exhibit pattern (\ref{snakey}).
\item Class [4] never exhibit pattern (\ref{snakev}).
\item Class [3-4] exhibit free variation.
\end{enumerate}

There is no obvious way to know which class a stem belongs to until
suffixation occurs. \citet{lockwood2017potawatomi} argues that the
class [4] pattern is coming to dominate but the situation is quite
complicated. There are cases of perfectly parallel [3-4] paradigms,
such as can be seen in Table \ref{bee}.

\begin{table}[H]
  \centering
  \begin{tabular}{l|ll}
             & [4] & [3] \\\hline
    {\sc pl} & amok & amoyek \\
    {\sc loc} & amoyek & amok \\
    {\sc obv} & amoyen & amon \\
    {\sc poss} & nde'amoyem & ndamom \\
  \end{tabular}
  \caption{{\sc amo}, `bee' exhibits completely parallel stem
    shapes. Derived from \citet[p.\ 86]{lockwood2017potawatomi}.}
  \label{bee}
\end{table}

Despite this, attested cases of perfect parallelism for all of the
four inflections given above are less frequent than mixtures. So the
lexeme exhibits a [3-4] pattern for only some of its paradigm,
possibly patterning with [3] or [4] for a portion. There is also
speaker variation.

\begin{table}[H]
  \centering
  \begin{tabular}{l|ll}
             & [4] & [3] \\\hline
    {\sc pl} & anmayek & \\
    {\sc loc} & & \\
    {\sc obv} & anmayin & \\
    {\sc poss} & & ndanmam \\
  \end{tabular}
  \caption{{\sc anma}, `German person' does not exhibits completely
    parallel stem shapes for speaker JT. Derived from \citet[p.\
    87]{lockwood2017potawatomi}.}
  \label{bee}
\end{table}

To the extent that we can reliably take the attested forms as an
indication that a [3] or [4] form is not acceptable, consider that not
all speakers find the past participle and past forms of {\sc dive} to
be equally good. Some find {\it dove} to sound odd as a past
participle.

\pex
\a He dived into the pool.
\a He dove into the pool.
\xe

\pex
\a He had dived into the pool.
\a He had dove into the pool.
\xe

Lockwood notes cases where there is no /y/ epenthesis, such as {\sc
  ogawa}, `walleye', which is validation of the existence of the [3]
class. He also claims the [4] class is developing as a kind of
default. Despite this, Lockwood's notion of stem classification would
completely collapse the [3] and [4] stems based on theoretical
criteria I feel remains unstated.

\paragraph{Takeaway} In Potawatomi there are two phonologically
conditioned options for suffixation of noun stems ending in
/a,ɛ:,o:/. One option results in /y/ epenthesis and the other not. For
some, but not all lexemes, some union of these behaviors is possible
for at least some forms. This is not a purely phonological
delineation. It is therefore lexical.

\subsubsection{Aside: Formal Class versus ``Real'' Classes}

In a formal analysis, I do not expect to capture all of the gradient
dimensions needed to accurately model such complex systems. My aim is
to have sufficient fidelity with respect to an actual natural language
pattern that it moves forward an understanding and investigation of
the phenomenon in question.

\subsection{Person Prefix}

Possessed Potawatomi nouns are prefixed to indicate the person of the
possessor. There are three primary patterns.

\begin{enumerate}
\item Inalienably possessed stems take a short form (Example
  (\ref{granpa}).
\item Stems beginning with /o/ (not /o:/) or /ə/ take a long form
  (Example (\ref{car}).
\item Other forms may take either.
\end{enumerate}

\pex
\a\label{granpa} {\it n}meshomes `my grandfather'
\a\label{car} {\it nd}odabyan `my car'
\a\label{metal} {\it n}biwabkom, {\it nde}biwabkom `my metal'
\xe

Note that inalienably possessed forms are lexically
idiosyncratic. Kinship and body part terms tend to be inalienably
possessed but some such terms are not inalienably possessed such as
`twin' in (\ref{twin}). There are terms that refer to common objects
that are inalienably possessed. Example (\ref{blanket}) shows that
terms with very similar meanings can be both inalienably possessed in
one case and non-inalienably possessed in another.

\ex\label{twin} nizhodé (`twin' (familial relation))\xe

\pex\label{blanket}
\a\label{blanket1} nmedné (`my blanket' inalienably possessed)
\a\label{blanket2} waboyan (`blanket')
\xe

The system was originally characterized by Hockett as a purely
phonologically conditioned system as in Table
\ref{prefixes}. Inalienably possessed stems tend to begin in
/a,ɛ:,o:/. Non-inalienably possessed stems that begin with these
vowels were said to have an underlying glottal stop at the
beginning. Therefore the overabundant forms were those that began with
a consonant. This is not currently true and there is some question
that it ever was a fact.\footnote{Though the inflected form for `bee',
  (where the apostrophe indicates a glottal stop) indicates this may
  have been the case. A long form with a glottal stop is one of the
  modern variants of the person prefix.}

\begin{table}[H]
\begin{center}
\begin{tabular}{lp{4cm}p{4cm}p{4cm}}
Person             & Before /o,ə/ (long form) & Before a Consonant & Before /a,ɛ:,o:/ (short form) \\
\hline
1                  & ned        & ned or n  & n            \\
2                  & ked        & ked or k  & k            \\
\end{tabular}
\caption{The Classic Description of Patterning of Prefixes}
\label{prefixes}
\end{center}
\end{table}

Rather than discuss the intricacies of the system in too much detail,
I will say that the use of a particular prefix is partially
conditioned by phonology and partially conditioned by semantic
criteria, though the class distinctions overall are ultimately
lexical.

Lockwood feels that long forms have been gaining ground as a default
but overabundance in person prefixes remains robust.

\paragraph{Takeaway} Just as for stem shape, there appear to be
classes that exhibit consistent behavior and then a large class that
exhibits the union of those behaviors.

\subsection{Possession Suffix}

The possession suffix is a form of redundant possession marking. There
are again three classes.

\begin{enumerate}
\item Inalienably possessed and some other nouns never take it
  (Example (\ref{granpa2})).
\item Some nouns always take it, particularly persons, people and
  animate plants (Example (\ref{chief})).
\item All other nouns occasionally take it (Example (\ref{string2})).
\end{enumerate}

\pex
\a\label{granpa2} nmeshomes `my grandfather'
\a\label{chief} ndogmam `my chief'
\a\label{string2} ndekobjegen, ndekobjegnem `my string'
\xe

\paragraph{Takeaway} Just as for stem shape and prefix, there appear
to be classes that exhibit consistent behavior and then a large class
that exhibits the union of those behaviors.

\section{Other Languages}

\begin{itemize}
\item There are patterns of prefix overabundance in Odawa
  \citep{odawagrammar}.
\item The suffixes exhibit an overabundant pattern in Odawa as well
  \citep{odawagrammar}. Though they are less redundant since there is
  no other marker of third person possession. They are also
  characterized as correlating to greater emphasis.
\item Plains Cree also exhibits overabundance with respect to the
  suffix, though it also sometimes indicates difference in meaning
  \citep{ahenakew1987cree}.
\end{itemize}

\section{Characteristics of a Formal Description}

The type of overabundance seen here is best captured by
multi-dominance hierarchies \citep{naranjo2016overabundance}. These
have been used in theories such as HPSG to represent inheritance
relationships \citep{flickinger1985structure, flickinger1987lexical,
  sag1987information} and more recently in Information-Based
Morphology for other types of variable phenomena
\citep{crysmann2012establishing, crysmann2016variable}.

Let us consider only the suffixes. The core of the class where the
possessive suffixes are not allowed are inalienably possessed
nouns. The core of the class where suffixes are required is made up of
terms for persons, animals and animate plants. The largest class, the
overabundant class, is the union of the behaviors of these two other
classes.

\begin{figure}[H]
  \centering
  \begin{tikzpicture}[>=stealth']
  \node (3) at (0,1) {inalienable};
  \node (34) at (3,0) {variable};
  \node (4) at (6,1) {animates};

  \draw[<-] (3) to (34);
  \draw[<-] (4) to (34);
\end{tikzpicture}

%%% Local Variables:
%%% mode: latex
%%% TeX-master: "../../morphology"
%%% End:

  \caption{A structure representing the class relationships that
    describe possessive suffix overabundance.}
  \label{dpoles}
\end{figure}

If each lexeme in Potawatomi is associated to a class, rules need only
be specified for {\it inalienable} and {\it animate} classes. Any
member of the variable class will already be a member of both of these
classes and do not require special rules.

\section{Comparison to Other Approaches
}
\begin{itemize}
\item \citet{stump2016inflectional} attempts a formal treatment of the
  {\it dreamed/dreamt} pair. He proposes that there are underlyingly
  two stems for {\sc dream}, one which is classified like regular verb
  stems and one, specific to the past tense, that classifies like {\sc
    mean}.
  \begin{itemize}
  \item A positive aspect of this approach is that it attempts to
    provide a unified analysis of overabundance, heteroclesis and
    suppletion.
  \item A negative is that it essentially calls for an analysis of
    overabundance involving homophonous suppletion, which is
    unparsimonious.
  \item It also fails to capture that {\sc dream} exhibits the union
    of two (semi-)productive behaviors. There is no unified class, all
    member exhibit two class related behaviors by on a case by case
    basis.
  \item The system strongly implies that one stem should be default
    but in a system like Potawatomi, there could logically be 8
    cellmates. Which should be default?
  \end{itemize}
\item \citep{naranjo2016overabundance} point out that
  Information-Based Morphology has the formal characteristics that
  allow it to handle overabundance similar to, but far more simple
  than, the Potawatomi case. I believe this is correct, though I am
  unaware of any fleshed-out analysis.
\item Though Network Morphology \citep{brown2012network} provides
  inheritance systems, as \citet{crysmann2016variable} point out, the
  system lacks multi-dominance hierarchies, which allow one to capture
  variable phenomena.
\end{itemize}

\section{What Overabundance Tells us About Grammar}

\subsection{Implications for Theory}

\begin{itemize}
\item Overabundance showcases the importance of analogical
  relationships in morphological systems. Simply mapping features to
  forms fails to capture the systematicity of overabundant systems
  such as Potawatomi.
\item Similarly, overabundance emphasizes that there is an
  organization of word-forms that is distinct from categories based on
  morphosyntactic features that define paradigm cells in many
  theories.
\item Controversially, I believe these two points severely undermine
  the realizational program.
\end{itemize}

\subsection{Implications for (Theories of) Development}

\begin{itemize}
\item How does overabundance exist together with non-overabundance?
  Why aren't all patterns positive evidence for active rules in the
  system? If we assume that kids shut off low frequency or spurious
  patterns in the course of their development, overabundance must
  require not only analogy with classes with consistent behavior but
  it must also require that members of the variable class occur
  frequently enough in all their forms that it reinforces the
  existence of the class, otherwise, whatever is making us choose that
  only one form of the past tense of {\sc drive}, {\sc hit}, {\sc
    mean} or {\sc walk} are correct, despite other possibilities,
  would act on {\sc dive} and {\sc dream} equally.  Note, this would
  mean that we cannot talk about systems such as Potawatomi possession
  suffixes in terms of only two classes. We must consider all three
  classes as providing evidence of the grammatical system.
\item Why doesn't indecision always result in defective paradigms?
  Part of this might be social but additionally, it may be the case
  that overabundance must be maintained. Examples of variability must
  occur with sufficient frequency for them to survive. Perhaps having
  larger variable classes helps, where more evidence of the
  variability is evident to a learner.
\item Why isn't there more overabundance? In addition to the need for
  maintenance mentioned above, social factors may be
  important.
\item How does overabundance interacts with over-regularization and
  developmental periods? Over-regularization happens earlier in
  development. Do we develop more of our overabundant behaviors at a
  later age when we exhibit greater frequency matching? Or, for
  something like Potawatomi, might it be the opposite. There is so
  much evidence of variability that variability, itself, is the
  regular rule?
\end{itemize}

%%% Local Variables:
%%% mode: latex
%%% TeX-master: "../morphology"
%%% End:
