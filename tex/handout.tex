
\section{Introduction}

\begin{itemize}
\item Potawatomi is a highly endangered Algonquian language spoken in
  Northern Wisconsin.
\item Since at least the 1930s, researchers have consistently
  documented overabundance in noun inflection
  \citep{hockett1948potawatomi2, potdict, lockwood2017potawatomi}.
\item Certain aspects of the overabundant system can be seen in other
  languages with genetic and areal relationships to Potawatomi
  \citep{odawagrammar, ahenakew1987cree}.
\end{itemize}

\section{What is Overabundance?}

\begin{itemize}
\item In the language of realizational morphology, overabundance
  \citep{hockett1947problems, thornton2011overabundance,
    thornton2012reduction} is when more than one form of a word can
  realize the same morphosyntactic properties.
\item Overabundant forms are called cellmates.
\end{itemize}

\pex
\a Last night I dreamt that somebody loved me. (Song lyric from {\it Last Night I dreamt that Somebody Loved Me} by the Smiths)
\a Last night I dreamed of chickens, (First line from the poem {\it Last Night I Dreamed of Chickens} by Jack Prelutsky)
\xe

\subsection{Hybrid-Class Overabundance}

\citet{naranjo2016overabundance} point out that one form that
overabundance \citep{hockett1947problems, thornton2011overabundance}
can take is \emph{hybrid-class overabundance}, where a lexeme belongs
to a class that exhibits the union of the behaviors of two or more
other classes.
        
\section{How is Potawatomi Overabundant?}

\begin{itemize}
\item Potawatomi noun inflection shows overabundant behavior according
  to three different criteria:
  \begin{enumerate}
  \item the form of prefixes indicating person (Example (\ref{canoe}))
    and
  \item the optionality of a suffix indicating possession (Example
    (\ref{string})).
  \item stem shape (Example (\ref{snake})),
  \end{enumerate}
\item The first two of these are predicted to affect the majority of
  nouns according to both \citet{hockett1948potawatomi2,
    lockwood2017potawatomi}.
\end{itemize}

\pex\label{canoe}
\a\label{canoeshort}
\begingl
\gla njiman // 
\glb n-jiman //
\glc {\sc 1.pos}-canoe //
\glft `my canoe' //
\endgl

\a\label{canoelong}
\begingl
\gla ndejiman //
\glb ned-jiman //
\glc {\sc 1.pos}-canoe //
\glft `my canoe' \trailingcitation{\citep[p. 69]{hockett1948potawatomi2}} //
\endgl
\xe

\pex\label{string}
\a\label{string-}
\begingl
\gla ndekobjegen // 
\glb n-tekopjegen //
\glc {\sc 1.pos}-string //
\glft `my string' //
\endgl

\a\label{string-m}
\begingl
\gla ndekobjegnem //
\glb n-tekopjegen-m //
\glc {\sc 1.pos}-string-{\sc poss} //
\glft `my string' \trailingcitation{\citep[p. 115]{potdict}} //
\endgl
\xe

\pex\label{snake}
\a\label{snakev}
\begingl
\gla mnedok // 
\glb mnedo-k //
\glc snake-{\sc pl} //
\glft `snakes' //
\endgl

\a\label{snakey}
\begingl
\gla mnedoyek //
\glb mnedoy-k //
\glc snake-{\sc pl} //
\glft `snakes' \trailingcitation{\citep[p. 80]{potdict}} //
\endgl
\xe

\subsection{Person Prefix}

Possessed Potawatomi nouns are prefixed to indicate the person of the
possessor. There are three primary patterns.

\begin{enumerate}
\item Inalienably possessed stems take a short form (Example
  (\ref{granpa}).
\item Stems beginning with /o/ (not /o:/) or /ə/ take a long form
  (Example (\ref{car}).
\item Other forms may take either.
\end{enumerate}

\pex
\a\label{granpa} {\it n}meshomes `my grandfather'
\a\label{car} {\it nd}odabyan `my car'
\a\label{metal} {\it n}biwabkom, {\it nde}biwabkom `my metal'
\xe

Note that inalienably possessed forms are lexically
idiosyncratic. Kinship and body part terms tend to be inalienably
possessed but some such terms are not inalienably possessed such as
`twin' in (\ref{twin}). There are terms that refer to common objects
that are inalienably possessed. Example (\ref{blanket}) shows that
terms with very similar meanings can be both inalienably possessed in
one case and non-inalienably possessed in another.

\ex\label{twin} nizhodé (`twin' (familial relation))\xe

\pex\label{blanket}
\a\label{blanket1} nmedné (`my blanket' inalienably possessed)
\a\label{blanket2} waboyan (`blanket')
\xe

The system was originally characterized by Hockett as a purely
phonologically conditioned system as in Table
\ref{prefixes}. Inalienably possessed stems tend to begin in
/a,ɛ:,o:/. Non-inalienably possessed stems that begin with these
vowels were said to have an underlying glottal stop at the
beginning. Therefore the overabundant forms were those that began with
a consonant. This is not currently true and there is some question
that it ever was a fact.\footnote{Though the inflected form for `bee',
  (where the apostrophe indicates a glottal stop) indicates this may
  have been the case. A long form with a glottal stop is one of the
  modern variants of the person prefix.}

\begin{table}[H]
\begin{center}
\begin{tabular}{lp{4cm}p{4cm}p{4cm}}
Person             & Before /o,ə/ (long form) & Before a Consonant & Before /a,ɛ:,o:/ (short form) \\
\hline
1                  & ned        & ned or n  & n            \\
2                  & ged        & ged or g  & g            \\
\end{tabular}
\caption{The Classic Description of Patterning of Prefixes}
\label{prefixes}
\end{center}
\end{table}

Rather than discuss the intricacies of the system in too much detail,
I will say that the use of a particular prefix is partially
conditioned by phonology and partially conditioned by semantic
criteria, though the class distinctions overall are ultimately
lexical.

Lockwood feels that long forms have been gaining ground as a default
but overabundance in person prefixes remains robust.

\paragraph{Takeaway} There appear to be classes that exhibit
consistent behavior and then a large class that exhibits the union of
those behaviors.

\subsection{Possession Suffix}

The possession suffix is a form of redundant possession marking. There
are again three classes.

\begin{enumerate}
\item Inalienably possessed and some other nouns never take it
  (Example (\ref{granpa2})).
\item Some nouns always take it, particularly persons, people and
  animate plants (Example (\ref{chief})).
\item All other nouns occasionally take it (Example (\ref{string2})).
\end{enumerate}

\pex
\a\label{granpa2} nmeshomes `my grandfather'
\a\label{chief} ndogmam `my chief'
\a\label{string2} ndekobjegen, ndekobjegnem `my string'
\xe

\paragraph{Takeaway} Just as for stem shape and prefix, there appear
to be classes that exhibit consistent behavior and then a large class
that exhibits the union of those behaviors.

\subsection{Stem Form Variation}

Hockett notes a number of stem classes in Potawatomi that can be
determined by various criteria. Three such classes are involved in the
variation seen in (\ref{snake}). Without a suffix, stems in these
classes end in /a,ɛ:,o:/. The classes are delineated according to
whether an epenthetic /y/ occurs between the stem and any suffix.

\begin{enumerate}
\item Class [3] never exhibit pattern (\ref{snakey}).
\item Class [4] never exhibit pattern (\ref{snakev}).
\item Class [3-4] exhibit free variation.
\end{enumerate}

There is no obvious way to know which class a stem belongs to until
suffixation occurs. \citet{lockwood2017potawatomi} argues that the
class [4] pattern is coming to dominate but the situation is quite
complicated. There are cases of perfectly parallel [3-4] paradigms,
such as can be seen in Table \ref{bee}.

\begin{table}[H]
  \centering
  \begin{tabular}{l|ll}
             & [4] & [3] \\\hline
    {\sc pl} & amok & amoyek \\
    {\sc loc} & amoyek & amok \\
    {\sc obv} & amoyen & amon \\
    {\sc poss} & nde'amoyem & ndamom \\
  \end{tabular}
  \caption{{\sc amo}, `bee' exhibits completely parallel stem
    shapes. Derived from \citet[p.\ 86]{lockwood2017potawatomi}.}
  \label{bee}
\end{table}

Despite this, attested cases of perfect parallelism for all of the
four inflections given above are less frequent than mixtures. So the
lexeme exhibits a [3-4] pattern for only some of its paradigm,
possibly patterning with [3] or [4] for a portion. There is also
speaker variation.

\begin{table}[H]
  \centering
  \begin{tabular}{l|ll}
             & [4] & [3] \\\hline
    {\sc pl} & anmayek & \\
    {\sc loc} & & \\
    {\sc obv} & anmayin & \\
    {\sc poss} & & ndanmam \\
  \end{tabular}
  \caption{{\sc anma}, `German person' does not exhibits completely
    parallel stem shapes for speaker JT. Derived from \citet[p.\
    87]{lockwood2017potawatomi}.}
  \label{bee}
\end{table}

To the extent that we can reliably take the attested forms as an
indication that a [3] or [4] form is not acceptable, consider that not
all speakers find the past participle and past forms of {\sc dive} to
be equally good. Some find {\it dove} to sound odd as a past
participle.

\pex
\a He dived into the pool.
\a He dove into the pool.
\xe

\pex
\a He had dived into the pool.
\a He had dove into the pool.
\xe

Lockwood notes cases where there is no /y/ epenthesis, such as {\sc
  ogawa}, `walleye', which is validation of the existence of the [3]
class. He also claims the [4] class is developing as a kind of
default. Despite this, Lockwood's notion of stem classification would
completely collapse the [3] and [4] stems based on theoretical
criteria I feel remains unstated.

\paragraph{Takeaway} In Potawatomi there are two phonologically
conditioned options for suffixation of noun stems ending in
/a,ɛ:,o:/. One option results in /y/ epenthesis and the other not. For
some, but not all lexemes, some union of these behaviors is possible
for at least some forms. This is not a purely phonological
delineation. It is therefore lexical.

\section{Other Languages}

Closely related languages exhibit similar patterns.

\begin{itemize}
\item There are patterns of prefix overabundance in Odawa
  \citep{odawagrammar}.
\item The suffixes exhibit an overabundant pattern in Odawa as well
  \citep{odawagrammar}. Though they are less redundant since there is
  no other marker of third person possession. They are also
  characterized as correlating to greater emphasis.
\item Plains Cree also exhibits overabundance with respect to the
  suffix, though it also sometimes indicates difference in meaning
  \citep{ahenakew1987cree}.
\end{itemize}

\section{Characteristics of a Formal Description}

Due to the fact that there appear to be classes that exhibit the union
of inflectional behaviors in the examples above, the patterns can be
characterized as hybrid-class overabundance.

\begin{itemize}
\item The approach taken here utilizes a non-inheritance-based
  relational approach using logical proofs.
\item It has similarities to systems used in \citet{lambek1997type,
    mcconville2006inheritance}.
\item The theory is separationist \citep{beard1995lexeme,
    aronoff1994morphology, sadler2001syntax}. Though it has a
  well-developed interface to the syntactic theory of Linear
  Categorial grammar \citep{smith2010correlational,
    mihalivcek2012distinguishing, mihalicek2012serbo,
    pollard2015agnostic, martin2013dynamics, martin2014dynamic,
    worth2016english, yasavul2017questions} only the purely
  morphological portion will be treated here.
\item The logic uses is a modern type theory
  \citep{martinlof1984intuitionistic}, the Calculus of Inductive
  Constructions \citep{coquand1988calculus, luo1990extended,
    coquand1990inductively}.
\item There is a computational implementation of a Potawatomi fragment
  for the theorem prover Coq ().
\end{itemize}

Let us consider only the suffixes. The core of the class where the
possessive suffixes are not allowed are inalienably possessed
nouns. The core of the class where suffixes are required is made up of
terms for persons, animals and animate plants. The largest class, the
overabundant class, is the union of the behaviors of these two other
classes.

\begin{figure}[H]
  \centering
  \begin{tikzpicture}[>=stealth']
  \node (3) at (0,1) {inalienable};
  \node (34) at (3,0) {variable};
  \node (4) at (6,1) {animates};

  \draw[<-] (3) to (34);
  \draw[<-] (4) to (34);
\end{tikzpicture}

%%% Local Variables:
%%% mode: latex
%%% TeX-master: "../../morphology"
%%% End:

  \caption{A structure representing the class relationships that
    describe possessive suffix overabundance.}
  \label{dpoles}
\end{figure}

If each lexeme in Potawatomi is associated to a class, rules need only
be specified for {\it inalienable} and {\it animate} classes. Any
member of the variable class will already be a member of both of these
classes and do not require special rules.

\section{Comparison to Other Approaches}

There are two approaches that I am aware of in the formal theoretical
literature, an inheritance-based approach and a multiple stem-based
approach.

\subsection{Problems for Multiple-Inheritance}

\citep{naranjo2016overabundance} suggest using multiple
inheritance systems to model this. These have been used in theories
such as HPSG \citep{flickinger1985structure, flickinger1987lexical,
  sag1987information} and DATR \citep{evans1996datr}.

\begin{enumerate}
\item Orthogonal multiple inheritance as seen in the DATR-based
  formalism of Network Morphology \citep{brown2012network} can only
  handle multiple inheritance where conflicts do not arise.
\item Information-Based Morphology (IbM)
  \citep{crysmann2012establishing, crysmann2016variable} uses a
  conflict mitigation strategy called online type construction
  \citep{koenig1994type}, which should allow them to avoid issues
  found in other multiple-inheritance systems.
\item It is not difficult to find data that causes problems for IbM,
  as well. There is a fundamental difficulty in avoiding conflicts
  when treating complex relational systems as data structure building.
\end{enumerate}

\subsection{Problems for Multiple Stems}

\citet{stump2016inflectional, bonami2016paradigm} attempt a formal
treatment of the {\it dreamed/dreamt} pair. They propose that there
are underlyingly two stems for {\sc dream}, one which is classified
like regular verb stems and one, specific to the past tense that
classifies like {\sc mean}.
  
\begin{itemize}
\item A positive aspect of this approach is that it attempts to
  provide a unified analysis of overabundance, heteroclesis and
  suppletion.
\item A negative is that it essentially calls for an analysis of
  overabundance involving homophonous suppletion, which is
  unparsimonious.
\item It also fails to capture that {\sc dream} exhibits the union of
  two inflectional behaviors. The descriptive generalization is not
  captured.
\item The system strongly implies that one stem should be default
  but in a system like Potawatomi, there could logically be 8
  cellmates. Which should be default?
\end{itemize}

%%% Local Variables:
%%% mode: latex
%%% TeX-master: "../potawatomiho"
%%% End:
